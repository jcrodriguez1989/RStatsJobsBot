\documentclass[]{elsarticle} %review=doublespace preprint=single 5p=2 column
%%% Begin My package additions %%%%%%%%%%%%%%%%%%%
\usepackage[hyphens]{url}

  \journal{Latinamerican Conference About the Use of R in
R\&D} % Sets Journal name


\usepackage{lineno} % add
\providecommand{\tightlist}{%
  \setlength{\itemsep}{0pt}\setlength{\parskip}{0pt}}

\usepackage{graphicx}
\usepackage{booktabs} % book-quality tables
%%%%%%%%%%%%%%%% end my additions to header

\usepackage[T1]{fontenc}
\usepackage{lmodern}
\usepackage{amssymb,amsmath}
\usepackage{ifxetex,ifluatex}
\usepackage{fixltx2e} % provides \textsubscript
% use upquote if available, for straight quotes in verbatim environments
\IfFileExists{upquote.sty}{\usepackage{upquote}}{}
\ifnum 0\ifxetex 1\fi\ifluatex 1\fi=0 % if pdftex
  \usepackage[utf8]{inputenc}
\else % if luatex or xelatex
  \usepackage{fontspec}
  \ifxetex
    \usepackage{xltxtra,xunicode}
  \fi
  \defaultfontfeatures{Mapping=tex-text,Scale=MatchLowercase}
  \newcommand{\euro}{€}
\fi
% use microtype if available
\IfFileExists{microtype.sty}{\usepackage{microtype}}{}
\bibliographystyle{elsarticle-harv}
\ifxetex
  \usepackage[setpagesize=false, % page size defined by xetex
              unicode=false, % unicode breaks when used with xetex
              xetex]{hyperref}
\else
  \usepackage[unicode=true]{hyperref}
\fi
\hypersetup{breaklinks=true,
            bookmarks=true,
            pdfauthor={},
            pdftitle={RStatsJobsBot: My journey on developing an R-based Twitter bot},
            colorlinks=false,
            urlcolor=blue,
            linkcolor=magenta,
            pdfborder={0 0 0}}
\urlstyle{same}  % don't use monospace font for urls

\setcounter{secnumdepth}{0}
% Pandoc toggle for numbering sections (defaults to be off)
\setcounter{secnumdepth}{0}

% Pandoc citation processing
\newlength{\cslhangindent}
\setlength{\cslhangindent}{1.5em}
\newlength{\csllabelwidth}
\setlength{\csllabelwidth}{3em}
% for Pandoc 2.8 to 2.10.1
\newenvironment{cslreferences}%
  {}%
  {\par}
% For Pandoc 2.11+
\newenvironment{CSLReferences}[2] % #1 hanging-ident, #2 entry spacing
 {% don't indent paragraphs
  \setlength{\parindent}{0pt}
  % turn on hanging indent if param 1 is 1
  \ifodd #1 \everypar{\setlength{\hangindent}{\cslhangindent}}\ignorespaces\fi
  % set entry spacing
  \ifnum #2 > 0
  \setlength{\parskip}{#2\baselineskip}
  \fi
 }%
 {}
\usepackage{calc}
\newcommand{\CSLBlock}[1]{#1\hfill\break}
\newcommand{\CSLLeftMargin}[1]{\parbox[t]{\csllabelwidth}{#1}}
\newcommand{\CSLRightInline}[1]{\parbox[t]{\linewidth - \csllabelwidth}{#1}\break}
\newcommand{\CSLIndent}[1]{\hspace{\cslhangindent}#1}

% Pandoc header



\begin{document}
\begin{frontmatter}

  \title{RStatsJobsBot: My journey on developing an R-based Twitter bot}
    \author[FAMAF - UNC]{Juan Cruz Rodriguez\corref{Corresponding
Author}}
   \ead{jcrodriguez@unc.edu.ar} 
      \address[FAMAF - UNC]{FAMAF, Universidad Nacional de Córdoba,
Argentina}
    
  \begin{abstract}
  
  \end{abstract}
   \begin{keyword} Automation, Github Actions, Free, Server\end{keyword}
 \end{frontmatter}

Twitter is one of the social networks most used by the R users
community. And possibly it is the social network that offers the
greatest flexibility for its programmatic access
(\href{https://github.com/ropensci/rtweet}{\texttt{\{rtweet\}}} (Kearney
2019)). In this regard, Twitter bots result as an excellent tool to
promote our product or tool. However, compared to other topics or
programming languages, it is not usual to find a wide variety of bots
related to R. This is not due to an increased difficulty itself, but
rather due to unfamiliarity with automated bots in R.

In this flash talk, I will show the learning path I took to bring an
idea to \href{https://twitter.com/RStatsJobsBot}{@RStatsJobsBot}, a
Twitter bot that currently has over 750 followers. The @RStatsJobsBot
runs entirely on R and is deployed and continuously running on Github
Actions. I intend that after this talk, \textbf{you will be able to
create your own R bot} without the need for additional hardware or a
server.

\hypertarget{references}{%
\section*{References}\label{references}}
\addcontentsline{toc}{section}{References}

\hypertarget{refs}{}
\begin{CSLReferences}{1}{0}
\leavevmode\hypertarget{ref-rtweet-package}{}%
Kearney, Michael W. 2019. {``Rtweet: Collecting and Analyzing Twitter
Data.''} \emph{Journal of Open Source Software} 4 (42): 1829.
\url{https://doi.org/10.21105/joss.01829}.

\end{CSLReferences}


\end{document}
